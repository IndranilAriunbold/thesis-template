\chapter{Conclusion and Future Work}
\label{ch:conclusion}
\graffito{Note: Zusammenfassung und Ausblick}

In conclusion, the proposed structure and operational framework provide a reliable AI-driven solution for controlling physical devices while ensuring adaptability, efficiency, and safety measures in real time. The integration of a modular design across components proposes not only a robust framework for this specific task but can also serve as a model for future AI applications in robotics and beyond.

The proposed LangGraph RAG ReAct framework for local deployment on a Raspberry Pi, augmented with ASR capabilities, represents a powerful tool in automating and controlling physical systems like a LEGO train. By leveraging the strengths of open-source software and hardware, the solution enhances user interaction through high-level reasoning capabilities and effective state management, ensuring safe and efficient operation. Future work may include expanding compatibility with additional devices and improving machine learning models for even more refined performance.

%
% Section: Summary
%
\section{Summary and Conclusion}
\label{sec:conclusion:summary}
\graffito{Note: Check}

%
% Section: Limitations
%
\section{Limitations}
\label{sec:conclusion:limitations}
\graffito{Note: Check}


%
% Section: Implementation
%
\section{Future Work}
\label{sec:conclusion:future_work}
\graffito{Note: Check}